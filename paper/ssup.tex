\documentclass{article} % For LaTeX2e
\usepackage{nips14submit_e,times}
\usepackage{hyperref}
\usepackage{url}
\usepackage{multirow}
%\documentstyle[nips14submit_09,times,art10]{article} % For LaTeX 2.09

\usepackage{framed}
\usepackage{algorithmic}
\usepackage{graphicx} 
\usepackage{amssymb}
\usepackage{amsmath}
\usepackage{amsthm}
\usepackage{caption}
\usepackage{subcaption}
\usepackage{listings}

\lstloadlanguages{Matlab}

\lstset{
  language=Matlab,
  frame=single,
  breaklines=true,
  basicstyle=\tiny,
  postbreak=\raisebox{0ex}[0ex][0ex]{\ensuremath{\color{red}\hookrightarrow\space}}
}

\newcommand{\script}[2]{
\lstinputlisting{#1}
}

\usepackage[compact]{titlesec}
\titlespacing{\section}{0pt}{0.5ex}{0.3ex}
\titlespacing{\subsection}{0pt}{0.2ex}{0ex}
\titlespacing{\subsubsection}{0pt}{0.1ex}{0ex}

\newcommand{\startcompact}[1]{\par\vspace{-0.75em}\begin{#1}%
  \allowdisplaybreaks\ignorespaces}

\newcommand{\stopcompact}[1]{\end{#1}\ignorespaces}

\usepackage{paralist}

\makeatletter
\ifcase \@ptsize \relax% 10pt
  \newcommand{\miniscule}{\@setfontsize\miniscule{4}{5}}% \tiny: 5/6
\or% 11pt
  \newcommand{\miniscule}{\@setfontsize\miniscule{5}{6}}% \tiny: 6/7
\or% 12pt
  \newcommand{\miniscule}{\@setfontsize\miniscule{5}{6}}% \tiny: 6/7
\fi
\makeatother

\newcommand {\aplt} {\ {\raise-.5ex\hbox{$\buildrel<\over\sim$}}\ }

\newcommand{\eqn}[1]{Eqn.~\ref{eqn:#1}}
\newcommand{\fig}[1]{Fig.~\ref{fig:#1}}
\newcommand{\tab}[1]{Table~\ref{tab:#1}}
\newcommand{\secc}[1]{Section~\ref{sec:#1}}
\def\etal{{\textit{et~al.~}}}
\newcommand{\BigO}[1]{\ensuremath{\operatorname{O}\left(#1\right)}}
\usepackage[symbol*]{footmisc}

\DefineFNsymbolsTM{myfnsymbols}{% def. from footmisc.sty "bringhurst" symbols
  \textasteriskcentered *
  \textdagger    \dagger
  \textdaggerdbl \ddagger
  \textsection   \mathsection
  \textbardbl    \|%
  \textparagraph \mathparagraph
}%

\nipsfinalcopy % Uncomment for camera-ready version

\begin{document}

\setcounter{section}{8}

\pagebreak
\section{Supplementary material}
We present the efficient expressions discovered by our system, using
Matlab-style syntax, and we visualize computation trees. Each example contains: (i) code that computes the
original target formulas; (ii) the formulae derived by our system and
(iii) code that verifies the correctness of the expression. The size of
matrices $n$, $m$ can be chosen arbitrary.

Code for generating the expressions can be downloaded from \url{https://github.com/kkurach/math_learning}.
The source files for this paper are available at \url{https://github.com/kkurach/math_learning/paper/}.




\subsection{$\mathbf{(\sum AA^T)_k}$}



{\bf k = 1}

\begin{lstlisting}
n = 100;
m = 200;
A = randn(n, m);
original = (sum((sum(A, 1)), 2));

optimized = 1 * ((sum(((sum((A'), 1))'), 1)));
normalization = sum(abs(original(:)));
assert(sum(abs(original(:) - optimized(:))) / normalization < 1e-10);
\end{lstlisting}


{\bf k = 2}

\begin{lstlisting}
n = 100;
m = 200;
A = randn(n, m);
original = (sum((sum((A * (A')), 1)), 2));

optimized = 1 * ((sum((A * ((sum(A, 1))')), 1)));
normalization = sum(abs(original(:)));
assert(sum(abs(original(:) - optimized(:))) / normalization < 1e-10);
\end{lstlisting}


{\bf k = 3}

\begin{lstlisting}
n = 100;
m = 200;
A = randn(n, m);
original = (sum((sum(((A * (A')) * A), 1)), 2));

optimized = 1 * ((sum((A * (((sum((A'), 1)) * A)')), 1)));
normalization = sum(abs(original(:)));
assert(sum(abs(original(:) - optimized(:))) / normalization < 1e-10);
\end{lstlisting}


{\bf k = 4}

\begin{lstlisting}
n = 100;
m = 200;
A = randn(n, m);
original = (sum((sum((((A * (A')) * A) * (A')), 1)), 2));

optimized = 1 * ((sum((A * ((((A * ((sum(A, 1))'))') * A)')), 1)));
normalization = sum(abs(original(:)));
assert(sum(abs(original(:) - optimized(:))) / normalization < 1e-10);
\end{lstlisting}


{\bf k = 5}

\begin{lstlisting}
n = 100;
m = 200;
A = randn(n, m);
original = (sum((sum(((((A * (A')) * A) * (A')) * A), 1)), 2));

optimized = 1 * ((sum((A * ((((A * (((sum((A'), 1)) * A)'))') * A)')), 1)));
normalization = sum(abs(original(:)));
assert(sum(abs(original(:) - optimized(:))) / normalization < 1e-10);
\end{lstlisting}


{\bf k = 6}

\begin{lstlisting}
n = 100;
m = 200;
A = randn(n, m);
original = (sum((sum((((((A * (A')) * A) * (A')) * A) * (A')), 1)), 2));

optimized = 1 * ((sum((A * ((((A * ((((A * ((sum(A, 1))'))') * A)'))') * A)')), 1)));
normalization = sum(abs(original(:)));
assert(sum(abs(original(:) - optimized(:))) / normalization < 1e-10);
\end{lstlisting}


{\bf k = 7}

\begin{lstlisting}
n = 100;
m = 200;
A = randn(n, m);
original = (sum((sum(((((((A * (A')) * A) * (A')) * A) * (A')) * A), 1)), 2));

optimized = 1 * ((sum((A * ((((A * ((((A * (((sum((A'), 1)) * A)'))') * A)'))') * A)')), 1)));
normalization = sum(abs(original(:)));
assert(sum(abs(original(:) - optimized(:))) / normalization < 1e-10);
\end{lstlisting}


{\bf k = 8}

\begin{lstlisting}
n = 100;
m = 200;
A = randn(n, m);
original = (sum((sum((((((((A * (A')) * A) * (A')) * A) * (A')) * A) * (A')), 1)), 2));

optimized = 1 * ((sum((A * ((((A * ((((A * ((((A * ((sum(A, 1))'))') * A)'))') * A)'))') * A)')), 1)));
normalization = sum(abs(original(:)));
assert(sum(abs(original(:) - optimized(:))) / normalization < 1e-10);
\end{lstlisting}


{\bf k = 9}

\begin{lstlisting}
n = 100;
m = 200;
A = randn(n, m);
original = (sum((sum(((((((((A * (A')) * A) * (A')) * A) * (A')) * A) * (A')) * A), 1)), 2));

optimized = 1 * ((sum((A * ((((A * ((((A * ((((A * (((sum((A'), 1)) * A)'))') * A)'))') * A)'))') * A)')), 1)));
normalization = sum(abs(original(:)));
assert(sum(abs(original(:) - optimized(:))) / normalization < 1e-10);
\end{lstlisting}


{\bf k = 10}

\begin{lstlisting}
n = 100;
m = 200;
A = randn(n, m);
original = (sum((sum((((((((((A * (A')) * A) * (A')) * A) * (A')) * A) * (A')) * A) * (A')), 1)), 2));

optimized = 1 * ((sum((A * ((((A * ((((A * ((((A * ((((A * ((sum(A, 1))'))') * A)'))') * A)'))') * A)'))') * A)')), 1)));
normalization = sum(abs(original(:)));
assert(sum(abs(original(:) - optimized(:))) / normalization < 1e-10);
\end{lstlisting}


{\bf k = 11}

\begin{lstlisting}
n = 100;
m = 200;
A = randn(n, m);
original = (sum((sum(((((((((((A * (A')) * A) * (A')) * A) * (A')) * A) * (A')) * A) * (A')) * A), 1)), 2));

optimized = 1 * ((sum((A * ((((A * ((((A * ((((A * ((((A * (((sum((A'), 1)) * A)'))') * A)'))') * A)'))') * A)'))') * A)')), 1)));
normalization = sum(abs(original(:)));
assert(sum(abs(original(:) - optimized(:))) / normalization < 1e-10);
\end{lstlisting}


{\bf k = 12}

\begin{lstlisting}
n = 100;
m = 200;
A = randn(n, m);
original = (sum((sum((((((((((((A * (A')) * A) * (A')) * A) * (A')) * A) * (A')) * A) * (A')) * A) * (A')), 1)), 2));

optimized = 1 * ((sum((A * ((((A * ((((A * ((((A * ((((A * ((((A * ((sum(A, 1))'))') * A)'))') * A)'))') * A)'))') * A)'))') * A)')), 1)));
normalization = sum(abs(original(:)));
assert(sum(abs(original(:) - optimized(:))) / normalization < 1e-10);
\end{lstlisting}


{\bf k = 13}

\begin{lstlisting}
n = 100;
m = 200;
A = randn(n, m);
original = (sum((sum(((((((((((((A * (A')) * A) * (A')) * A) * (A')) * A) * (A')) * A) * (A')) * A) * (A')) * A), 1)), 2));

optimized = 1 * ((sum((A * ((((A * ((((A * ((((A * ((((A * ((((A * (((sum((A'), 1)) * A)'))') * A)'))') * A)'))') * A)'))') * A)'))') * A)')), 1)));
normalization = sum(abs(original(:)));
assert(sum(abs(original(:) - optimized(:))) / normalization < 1e-10);
\end{lstlisting}


{\bf k = 14}

\begin{lstlisting}
n = 100;
m = 200;
A = randn(n, m);
original = (sum((sum((((((((((((((A * (A')) * A) * (A')) * A) * (A')) * A) * (A')) * A) * (A')) * A) * (A')) * A) * (A')), 1)), 2));

optimized = 1 * ((sum((A * ((((A * ((((A * ((((A * ((((A * ((((A * ((((A * ((sum(A, 1))'))') * A)'))') * A)'))') * A)'))') * A)'))') * A)'))') * A)')), 1)));
normalization = sum(abs(original(:)));
assert(sum(abs(original(:) - optimized(:))) / normalization < 1e-10);
\end{lstlisting}


\subsection{$\mathbf{(\sum AB)_k}$}



{\bf k = 1}

\begin{lstlisting}
n = 100;
m = 200;
A = randn(n, m);
B = randn(m, n);
original = (sum((sum(A, 1)), 2));

optimized = 1 * ((sum((sum(A, 1)), 2)));
normalization = sum(abs(original(:)));
assert(sum(abs(original(:) - optimized(:))) / normalization < 1e-10);
\end{lstlisting}


{\bf k = 2}

\begin{lstlisting}
n = 100;
m = 200;
A = randn(n, m);
B = randn(m, n);
original = (sum((sum((A * B), 1)), 2));

optimized = 1 * (((sum(A, 1)) * (sum(B, 2))));
normalization = sum(abs(original(:)));
assert(sum(abs(original(:) - optimized(:))) / normalization < 1e-10);
\end{lstlisting}


{\bf k = 3}

\begin{lstlisting}
n = 100;
m = 200;
A = randn(n, m);
B = randn(m, n);
original = (sum((sum(((A * B) * A), 1)), 2));

optimized = 1 * ((sum((((sum(A, 1)) * B) * A), 2)));
normalization = sum(abs(original(:)));
assert(sum(abs(original(:) - optimized(:))) / normalization < 1e-10);
\end{lstlisting}


{\bf k = 4}

\begin{lstlisting}
n = 100;
m = 200;
A = randn(n, m);
B = randn(m, n);
original = (sum((sum((((A * B) * A) * B), 1)), 2));

optimized = 1 * ((sum(((((sum(A, 1)) * B) * A) * B), 2)));
normalization = sum(abs(original(:)));
assert(sum(abs(original(:) - optimized(:))) / normalization < 1e-10);
\end{lstlisting}


{\bf k = 5}

\begin{lstlisting}
n = 100;
m = 200;
A = randn(n, m);
B = randn(m, n);
original = (sum((sum(((((A * B) * A) * B) * A), 1)), 2));

optimized = 1 * ((sum((((((sum(A, 1)) * B) * A) * B) * A), 2)));
normalization = sum(abs(original(:)));
assert(sum(abs(original(:) - optimized(:))) / normalization < 1e-10);
\end{lstlisting}


{\bf k = 6}

\begin{lstlisting}
n = 100;
m = 200;
A = randn(n, m);
B = randn(m, n);
original = (sum((sum((((((A * B) * A) * B) * A) * B), 1)), 2));

optimized = 1 * ((sum(((((((sum(A, 1)) * B) * A) * B) * A) * B), 2)));
normalization = sum(abs(original(:)));
assert(sum(abs(original(:) - optimized(:))) / normalization < 1e-10);
\end{lstlisting}


{\bf k = 7}

\begin{lstlisting}
n = 100;
m = 200;
A = randn(n, m);
B = randn(m, n);
original = (sum((sum(((((((A * B) * A) * B) * A) * B) * A), 1)), 2));

optimized = 1 * ((sum((((((((sum(A, 1)) * B) * A) * B) * A) * B) * A), 2)));
normalization = sum(abs(original(:)));
assert(sum(abs(original(:) - optimized(:))) / normalization < 1e-10);
\end{lstlisting}


{\bf k = 8}

\begin{lstlisting}
n = 100;
m = 200;
A = randn(n, m);
B = randn(m, n);
original = (sum((sum((((((((A * B) * A) * B) * A) * B) * A) * B), 1)), 2));

optimized = 1 * ((sum(((((((((sum(A, 1)) * B) * A) * B) * A) * B) * A) * B), 2)));
normalization = sum(abs(original(:)));
assert(sum(abs(original(:) - optimized(:))) / normalization < 1e-10);
\end{lstlisting}


{\bf k = 9}

\begin{lstlisting}
n = 100;
m = 200;
A = randn(n, m);
B = randn(m, n);
original = (sum((sum(((((((((A * B) * A) * B) * A) * B) * A) * B) * A), 1)), 2));

optimized = 1 * ((sum((((((((((sum(A, 1)) * B) * A) * B) * A) * B) * A) * B) * A), 2)));
normalization = sum(abs(original(:)));
assert(sum(abs(original(:) - optimized(:))) / normalization < 1e-10);
\end{lstlisting}


{\bf k = 10}

\begin{lstlisting}
n = 100;
m = 200;
A = randn(n, m);
B = randn(m, n);
original = (sum((sum((((((((((A * B) * A) * B) * A) * B) * A) * B) * A) * B), 1)), 2));

optimized = 1 * ((sum(((((((((((sum(A, 1)) * B) * A) * B) * A) * B) * A) * B) * A) * B), 2)));
normalization = sum(abs(original(:)));
assert(sum(abs(original(:) - optimized(:))) / normalization < 1e-10);
\end{lstlisting}


{\bf k = 11}

\begin{lstlisting}
n = 100;
m = 200;
A = randn(n, m);
B = randn(m, n);
original = (sum((sum(((((((((((A * B) * A) * B) * A) * B) * A) * B) * A) * B) * A), 1)), 2));

optimized = 1 * ((sum((((((((((((sum(A, 1)) * B) * A) * B) * A) * B) * A) * B) * A) * B) * A), 2)));
normalization = sum(abs(original(:)));
assert(sum(abs(original(:) - optimized(:))) / normalization < 1e-10);
\end{lstlisting}


{\bf k = 12}

\begin{lstlisting}
n = 100;
m = 200;
A = randn(n, m);
B = randn(m, n);
original = (sum((sum((((((((((((A * B) * A) * B) * A) * B) * A) * B) * A) * B) * A) * B), 1)), 2));

optimized = 1 * ((sum(((((((((((((sum(A, 1)) * B) * A) * B) * A) * B) * A) * B) * A) * B) * A) * B), 2)));
normalization = sum(abs(original(:)));
assert(sum(abs(original(:) - optimized(:))) / normalization < 1e-10);
\end{lstlisting}


{\bf k = 13}

\begin{lstlisting}
n = 100;
m = 200;
A = randn(n, m);
B = randn(m, n);
original = (sum((sum(((((((((((((A * B) * A) * B) * A) * B) * A) * B) * A) * B) * A) * B) * A), 1)), 2));

optimized = 1 * ((sum((((((((((((((sum(A, 1)) * B) * A) * B) * A) * B) * A) * B) * A) * B) * A) * B) * A), 2)));
normalization = sum(abs(original(:)));
assert(sum(abs(original(:) - optimized(:))) / normalization < 1e-10);
\end{lstlisting}


{\bf k = 14}

\begin{lstlisting}
n = 100;
m = 200;
A = randn(n, m);
B = randn(m, n);
original = (sum((sum((((((((((((((A * B) * A) * B) * A) * B) * A) * B) * A) * B) * A) * B) * A) * B), 1)), 2));

optimized = 1 * ((sum(((((((((((((((sum(A, 1)) * B) * A) * B) * A) * B) * A) * B) * A) * B) * A) * B) * A) * B), 2)));
normalization = sum(abs(original(:)));
assert(sum(abs(original(:) - optimized(:))) / normalization < 1e-10);
\end{lstlisting}


{\bf k = 15}

\begin{lstlisting}
n = 100;
m = 200;
A = randn(n, m);
B = randn(m, n);
original = (sum((sum(((((((((((((((A * B) * A) * B) * A) * B) * A) * B) * A) * B) * A) * B) * A) * B) * A), 1)), 2));

optimized = 1 * ((sum((((((((((((((((sum(A, 1)) * B) * A) * B) * A) * B) * A) * B) * A) * B) * A) * B) * A) * B) * A), 2)));
normalization = sum(abs(original(:)));
assert(sum(abs(original(:) - optimized(:))) / normalization < 1e-10);
\end{lstlisting}


\subsection{$\mathbf{(\sum (A.*A)A^T})_k$}



{\bf k = 1}

\begin{lstlisting}
n = 100;
m = 200;
A = randn(n, m);
original = (sum((sum(A, 1)), 2));

optimized = 1 * ((sum((sum(A, 2)), 1)));
normalization = sum(abs(original(:)));
assert(sum(abs(original(:) - optimized(:))) / normalization < 1e-10);
\end{lstlisting}


{\bf k = 2}

\begin{lstlisting}
n = 100;
m = 200;
A = randn(n, m);
original = (sum((sum((A * (A')), 1)), 2));

optimized = 1 * ((sum(((sum(A, 1)) .* (sum(A, 1))), 2)));
normalization = sum(abs(original(:)));
assert(sum(abs(original(:) - optimized(:))) / normalization < 1e-10);
\end{lstlisting}


{\bf k = 3}

\begin{lstlisting}
n = 100;
m = 200;
A = randn(n, m);
original = (sum((sum(((A * (A')) * (A .* A)), 1)), 2));

optimized = 1 * ((sum((sum(((repmat((sum((repmat((sum(A, 1)), n, 1) .* A), 2)), 1, m) .* A) .* A), 2)), 1)));
normalization = sum(abs(original(:)));
assert(sum(abs(original(:) - optimized(:))) / normalization < 1e-10);
\end{lstlisting}


{\bf k = 4}

\begin{lstlisting}
n = 100;
m = 200;
A = randn(n, m);
original = (sum((sum((((A * (A')) * (A .* A)) * (A')), 1)), 2));

optimized = 1 * ((sum((sum((repmat((sum((repmat((sum(((repmat((sum(A, 1)), n, 1) .* A) .* A), 2)), 1, m) .* A), 1)), n, 1) .* A), 2)), 1)));
normalization = sum(abs(original(:)));
assert(sum(abs(original(:) - optimized(:))) / normalization < 1e-10);
\end{lstlisting}


\subsection{{\bf Sym$_k$}}



{\bf k = 1}

\begin{lstlisting}
n = 1;
m = 18;
A = randn(1, m);
sub = nchoosek(1:m, 1);
original = 0;
for i = 1:size(sub, 1)
  original = original + prod(A(sub(i, :)));
end

optimized = 1 * ((sum(A, 2)));
normalization = sum(abs(original(:)));
assert(sum(abs(original(:) - optimized(:))) / normalization < 1e-10);
\end{lstlisting}


{\bf k = 2}

\begin{lstlisting}
n = 1;
m = 18;
A = randn(1, m);
sub = nchoosek(1:m, 2);
original = 0;
for i = 1:size(sub, 1)
  original = original + prod(A(sub(i, :)));
end

optimized = (1 / 2) * (((sum(A, 2)) * (sum(A, 2)))) + (50 / -100) * ((A * (A')));
normalization = sum(abs(original(:)));
assert(sum(abs(original(:) - optimized(:))) / normalization < 1e-10);
\end{lstlisting}


{\bf k = 3}

\begin{lstlisting}
n = 1;
m = 18;
A = randn(1, m);
sub = nchoosek(1:m, 3);
original = 0;
for i = 1:size(sub, 1)
  original = original + prod(A(sub(i, :)));
end

optimized = (1 / 6) * ((sum(((sum(A, 2)) * ((sum(A, 2)) * A)), 2))) + (50 / -100) * ((A * (((sum(A, 2)) * A)'))) + (1 / 3) * (((A .* A) * (A')));
normalization = sum(abs(original(:)));
assert(sum(abs(original(:) - optimized(:))) / normalization < 1e-10);
\end{lstlisting}


{\bf k = 4}

\begin{lstlisting}
n = 1;
m = 18;
A = randn(1, m);
sub = nchoosek(1:m, 4);
original = 0;
for i = 1:size(sub, 1)
  original = original + prod(A(sub(i, :)));
end

optimized = (25 / -100) * ((A * (((sum(A, 2)) * ((sum(A, 2)) * A))'))) + (1 / 8) * ((A * ((A * ((A') * A))'))) + (1 / 3) * (((A * ((A') .* (A'))) * (sum(A, 2)))) + (25 / -100) * (((((A') .* (A'))') * ((A') .* (A')))) + (1 / 24) * ((sum(((sum(((sum(A, 2)) * ((sum(A, 2)) * A)), 2)) * A), 2)));
normalization = sum(abs(original(:)));
assert(sum(abs(original(:) - optimized(:))) / normalization < 1e-10);
\end{lstlisting}


\subsection{{\bf (RBM-1)$_k$}}



{\bf k = 1}

\begin{lstlisting}
n = 14;
m = 1;
A = randn(1, n);
nset = dec2bin(0:(2^(n) - 1));
original = 0;
for i = 1:size(nset, 1)
  v = logical(nset(i, :) - '0');
  original = original + (v * A') ^ 1;
end

optimized = 2^(n - 3) * (4 * ((sum((A'), 1))));
normalization = sum(abs(original(:)));
assert(sum(abs(original(:) - optimized(:))) / normalization < 1e-10);
\end{lstlisting}


{\bf k = 2}

\begin{lstlisting}
n = 14;
m = 1;
A = randn(1, n);
nset = dec2bin(0:(2^(n) - 1));
original = 0;
for i = 1:size(nset, 1)
  v = logical(nset(i, :) - '0');
  original = original + (v * A') ^ 2;
end

optimized = 2^(n - 3) * (2 * ((sum(((A .* A)'), 1))) + 2 * ((sum(((A') .* repmat((sum((A'), 1)), m, 1)), 1))));
normalization = sum(abs(original(:)));
assert(sum(abs(original(:) - optimized(:))) / normalization < 1e-10);
\end{lstlisting}


{\bf k = 3}

\begin{lstlisting}
n = 14;
m = 1;
A = randn(1, n);
nset = dec2bin(0:(2^(n) - 1));
original = 0;
for i = 1:size(nset, 1)
  v = logical(nset(i, :) - '0');
  original = original + (v * A') ^ 3;
end

optimized = 2^(n - 4) * (6 * ((sum((((((A') .* repmat((sum((A'), 1)), m, 1))') .* A)'), 1))) + 2 * ((sum(((A') .* repmat((sum(((A') .* repmat((sum((A'), 1)), m, 1)), 1)), m, 1)), 1))));
normalization = sum(abs(original(:)));
assert(sum(abs(original(:) - optimized(:))) / normalization < 1e-10);
\end{lstlisting}


{\bf k = 4}

\begin{lstlisting}
n = 14;
m = 1;
A = randn(1, n);
nset = dec2bin(0:(2^(n) - 1));
original = 0;
for i = 1:size(nset, 1)
  v = logical(nset(i, :) - '0');
  original = original + (v * A') ^ 4;
end

optimized = 2^(n - 5) * (12 * (((sum(((A .* A)'), 1)) .* (sum(((A') .* repmat((sum((A'), 1)), m, 1)), 1)))) + 6 * (((sum(((A .* A)'), 1)) .* (sum(((A .* A)'), 1)))) + 2 * ((sum(((A') .* repmat((sum(((A') .* repmat((sum(((A') .* repmat((sum((A'), 1)), m, 1)), 1)), m, 1)), 1)), m, 1)), 1))) + -4 * ((sum(((((((A .* A)') .* (A'))') .* A)'), 1))));
normalization = sum(abs(original(:)));
assert(sum(abs(original(:) - optimized(:))) / normalization < 1e-10);
\end{lstlisting}


{\bf k = 5}

\begin{lstlisting}
n = 14;
m = 1;
A = randn(1, n);
nset = dec2bin(0:(2^(n) - 1));
original = 0;
for i = 1:size(nset, 1)
  v = logical(nset(i, :) - '0');
  original = original + (v * A') ^ 5;
end

optimized = 2^(n - 6) * (30 * (((sum(((A .* A)'), 1)) .* (sum((((((A') .* repmat((sum((A'), 1)), m, 1))') .* A)'), 1)))) + 20 * (((sum(((A .* A)'), 1)) .* (sum(((A') .* repmat((sum(((A') .* repmat((sum((A'), 1)), m, 1)), 1)), m, 1)), 1)))) + -20 * ((sum(((A') .* repmat((sum(((((((A .* A)') .* (A'))') .* A)'), 1)), m, 1)), 1))) + 2 * ((sum(((A') .* repmat((sum(((A') .* repmat((sum(((A') .* repmat((sum(((A') .* repmat((sum((A'), 1)), m, 1)), 1)), m, 1)), 1)), m, 1)), 1)), m, 1)), 1))));
normalization = sum(abs(original(:)));
assert(sum(abs(original(:) - optimized(:))) / normalization < 1e-10);
\end{lstlisting}


{\bf k = 6}

\begin{lstlisting}
n = 14;
m = 1;
A = randn(1, n);
nset = dec2bin(0:(2^(n) - 1));
original = 0;
for i = 1:size(nset, 1)
  v = logical(nset(i, :) - '0');
  original = original + (v * A') ^ 6;
end

optimized = 2^(n - 7) * (30 * ((sum(((((((A') .* repmat((sum(((A .* A)'), 1)), m, 1))') .* repmat((sum(((A .* A)'), 1)), 1, m))') .* (A')), 1))) + -60 * (((sum(((((((A .* A)') .* (A'))') .* A)'), 1)) .* (sum(((A .* A)'), 1)))) + 90 * ((sum((((((A') .* repmat((sum(((A .* A)'), 1)), m, 1))') .* repmat((sum(((A') .* repmat((sum(((A .* A)'), 1)), m, 1)), 1)), 1, m))'), 1))) + -60 * ((sum(((A') .* repmat((sum(((A') .* repmat((sum(((((((A .* A)') .* (A'))') .* A)'), 1)), m, 1)), 1)), m, 1)), 1))) + 2 * ((sum(((A') .* repmat((sum(((A') .* repmat((sum(((A') .* repmat((sum(((A') .* repmat((sum(((A') .* repmat((sum((A'), 1)), m, 1)), 1)), m, 1)), 1)), m, 1)), 1)), m, 1)), 1)), m, 1)), 1))) + 30 * ((sum(((A') .* repmat((sum(((A') .* repmat((sum(((A') .* repmat((sum((((((A') .* repmat((sum((A'), 1)), m, 1))') .* A)'), 1)), m, 1)), 1)), m, 1)), 1)), m, 1)), 1))) + 32 * ((sum(((((((((((A .* A)') .* (A'))') .* A)') .* (A'))') .* A)'), 1))));
normalization = sum(abs(original(:)));
assert(sum(abs(original(:) - optimized(:))) / normalization < 1e-10);
\end{lstlisting}


\subsection{{\bf (RBM-2)$_k$}}



{\bf k = 1}

\begin{lstlisting}
n = 7;
m = 8;
A = randn(n, m);
nset = dec2bin(0:(2^(n) - 1));
mset = dec2bin(0:(2^(m) - 1));
original = 0;
for i = 1:size(nset, 1)
  for j = 1:size(mset, 1)
    v = logical(nset(i, :) - '0');
    h = logical(mset(j, :) - '0');
    original = original + (v * A * h') ^ 1;
  end
end

optimized = 2^(n + m - 5) * (8 * ((sum(((sum(A, 1))'), 1))));
normalization = sum(abs(original(:)));
assert(sum(abs(original(:) - optimized(:))) / normalization < 1e-10);
\end{lstlisting}


{\bf k = 2}

\begin{lstlisting}
n = 7;
m = 8;
A = randn(n, m);
nset = dec2bin(0:(2^(n) - 1));
mset = dec2bin(0:(2^(m) - 1));
original = 0;
for i = 1:size(nset, 1)
  for j = 1:size(mset, 1)
    v = logical(nset(i, :) - '0');
    h = logical(mset(j, :) - '0');
    original = original + (v * A * h') ^ 2;
  end
end

optimized = 2^(n + m - 5) * (2 * ((sum((sum((((sum(A, 2)) * (sum(A, 1)))'), 1)), 2))) + 2 * ((((sum(A, 2))') * (sum(A, 2)))) + 2 * (((sum(A, 1)) * ((sum(A, 1))'))) + 2 * ((sum(((sum((A .* A), 1))'), 1))));
normalization = sum(abs(original(:)));
assert(sum(abs(original(:) - optimized(:))) / normalization < 1e-10);
\end{lstlisting}


{\bf k = 3}

\begin{lstlisting}
n = 7;
m = 8;
A = randn(n, m);
nset = dec2bin(0:(2^(n) - 1));
mset = dec2bin(0:(2^(m) - 1));
original = 0;
for i = 1:size(nset, 1)
  for j = 1:size(mset, 1)
    v = logical(nset(i, :) - '0');
    h = logical(mset(j, :) - '0');
    original = original + (v * A * h') ^ 3;
  end
end

optimized = 2^(n + m - 7) * (12 * ((sum((((A * ((sum(A, 1))'))') * A), 2))) + 2 * ((sum(((((sum(((sum(A, 1))'), 1)) * (sum(A, 1)))') * (sum((sum(A, 1)), 2))), 1))) + 6 * ((sum(((sum(((repmat((sum(((sum(A, 1))'), 1)), n, m) .* A) .* A), 1))'), 1))) + 6 * ((sum((sum(((((sum(A, 1))') * (sum((A'), 1))) * A), 1)), 2))) + 6 * (((sum((sum(A, 1)), 2)) * ((sum(A, 1)) * ((sum(A, 1))')))));
normalization = sum(abs(original(:)));
assert(sum(abs(original(:) - optimized(:))) / normalization < 1e-10);
\end{lstlisting}


{\bf k = 4}

\begin{lstlisting}
n = 7;
m = 8;
A = randn(n, m);
nset = dec2bin(0:(2^(n) - 1));
mset = dec2bin(0:(2^(m) - 1));
original = 0;
for i = 1:size(nset, 1)
  for j = 1:size(mset, 1)
    v = logical(nset(i, :) - '0');
    h = logical(mset(j, :) - '0');
    original = original + (v * A * h') ^ 4;
  end
end

optimized = 2^(n + m - 9) * (48 * (((sum(A, 1)) * ((sum(((((sum(A, 2)) * (sum(A, 1)))') * A), 1))'))) + -4 * (((sum((repmat((sum(A, 1)), n, 1) .* A), 1)) * ((sum((repmat((sum(A, 1)), n, 1) .* A), 1))'))) + 8 * ((sum((sum((((A .* A) .* A) .* A), 1)), 2))) + 6 * ((((((sum(A, 2))') * (sum(A, 2))) * ((sum(A, 2))')) * (sum(A, 2)))) + -4 * (((sum((((((A') .* repmat(((sum(A, 2))'), m, 1))') .* repmat((sum(A, 2)), 1, m))'), 1)) * (sum(A, 2)))) + 6 * (((sum(A, 1)) * ((sum(((A * ((sum(A, 1))')) * (sum(A, 1))), 1))'))) + 12 * (((sum((sum(A, 2)), 1)) * (sum((sum(((((sum(A, 2)) * (sum(A, 1)))') * A), 2)), 1)))) + -12 * (((sum(((A .* A)'), 1)) * ((sum(((A .* A)'), 1))'))) + 24 * (((sum(A, 1)) * ((((A * ((sum(A, 1))'))') * A)'))) + 12 * (((sum((sum(A, 2)), 1)) * ((sum((sum(A, 1)), 2)) * ((sum(A, 1)) * ((sum(A, 1))'))))) + 2 * (((sum((sum((((sum(A, 2)) * (sum(A, 1)))'), 1)), 2)) * (sum((sum((((sum(A, 2)) * (sum(A, 1)))'), 1)), 2)))) + 6 * ((sum((sum(((repmat((sum(((sum((A .* A), 1))'), 1)), n, m) .* A) .* A), 1)), 2))) + 12 * (((sum((sum(A, 2)), 1)) * (sum((sum(((repmat((sum(((sum(A, 1))'), 1)), n, m) .* A) .* A), 1)), 2)))) + 12 * ((sum((sum(((A * ((sum(A, 1))')) * (sum((A .* A), 1))), 1)), 2))) + -24 * (((sum(A, 1)) * ((sum(((repmat((sum(A, 1)), n, 1) .* A) .* A), 1))'))) + 12 * (((sum((sum((A .* A), 1)), 2)) * (((sum(A, 2))') * (sum(A, 2))))) + 12 * (((sum(A, 1)) * (sum(((((sum(A, 2)) * (sum(A, 1)))') * A), 2)))) + 24 * (((((sum(A, 2))') * A) * ((((sum(A, 2))') * A)'))) + 12 * ((sum(((sum((((A') * A) .* ((A') * A)), 1))'), 1))) + -12 * (((sum((A .* A), 1)) * ((sum((A .* A), 1))'))) + -24 * (((sum(((A') .* repmat(((sum(A, 2))'), m, 1)), 1)) * (sum((A .* A), 2)))));
normalization = sum(abs(original(:)));
assert(sum(abs(original(:) - optimized(:))) / normalization < 1e-10);
\end{lstlisting}


\end{document}


